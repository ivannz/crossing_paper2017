\documentclass[a4paper]{article}
\usepackage[utf8]{inputenc}

\usepackage{graphicx, url}

\usepackage{amsmath, amsfonts, amssymb, amsthm}
\usepackage{xfrac, mathptmx}

\newcommand{\Real}{\mathbb{R}}
\newcommand{\Cplx}{\mathbb{C}}
\newcommand{\Pwr}{\mathcal{P}}
\newcommand{\pr}{\mathbb{P}}
\newcommand{\ex}{\mathbb{E}}

\usepackage[english, russian]{babel}
\newcommand{\eng}[1]{\foreignlanguage{english}{#1}}
\newcommand{\rus}[1]{\foreignlanguage{russian}{#1}}

\title{Intuition behind the scaling of durations}
\author{Nazarov Ivan, \rus{101мНОД(ИССА)}\\the DataScience Collective}
\begin{document}
\maketitle

\noindent Let $\{X(t)\}_{t\geq0}$ be a continuous $H$-sssi process. Define the crossing
duration as
\[ w_n(t) = \inf\bigl\{ s\geq t : |X(s) - X(t)| \geq 2^n \delta  \bigr\} - t\,, \]
for some $\delta > 0$ and $t\geq 0$. By stationarity of increments and
self-similarity of $\{X(t)\}$ for any $n\geq 0$:
\[
X(t + w_n) - X(t) \underset{\text{si}}{\overset{\mathcal{D}}{\sim}}
X(w_n) \underset{\text{ss}}{\overset{\mathcal{D}}{\sim}}
w_n^H X(1) \,.
\]
(\emph{do ss ans si generalise to stopping times?}).

Since $\{X(t)\}$ is continuous
\[ X(t+w_n)-X(t) = \delta 2^n \text{ a.s}\,,\]
whence almost surely
\[ \frac{1}{\delta 2^n}\bigl( X(t+w_n)-X(t)\bigr) = \frac{1}{\delta} \bigl( X(t+w_0) - X(t) \bigr)\,.\]
Thus we get
\[ \frac{1}{\delta} 2^{-n} w_n^H X(1) \overset{\mathcal{D}}{\sim} \frac{1}{\delta} w_0^H X(1) \,, \]
which implies that (\emph{not sure here as well...})
\[ 2^{-n} w_n^H \overset{\mathcal{D}}{\sim} w_0^H \] 
and
\[ w_0 \overset{\mathcal{D}}{\sim} 4^{-\frac{n}{2H}} w_n \] 

\end{document}
