\chapter{Conclusion and further work} % (fold)
\label{cha:conlusion_and_further_work}

This preformed extensive Monte-Carlo simulation suggests, that the considered
classes of $H$-SSSI stochastic processes indeed shares common statistical properties
of the crossing tree. To findings are summarized below: \begin{enumerate}
	\item The crossing tree seems capable of correctly identifying the Hurst
	exponent of the studied series, provided correct tree levels are chosen to
	produce the estimate. The tree levels should not be too close to the leaves
	of the tree, for the sake of avoiding excessive bias due to linear interpolation,
	and be such that there is no statistical evidence to significant variations
	in the offspring distribution at each level;
	\item Processes seem to share similar shape of the empirical distribution of
	the crossing sizes (the number of offspring), though the hypothesized
	relationship between the offspring distribution and the Hurst exponent
	does not seem to hold exactly, (see figures \ref{fig:fbm_offspring_distribution}
	and \ref{fig:all_xing_probs});
	\item The conditional distributions of excursion within each crossing seems
	to agree with the hypothesised values of $2^{\sfrac{-1}{2H}}$.
	\item The collected empirical evidence seems to confirm similarity of the scaling
	properties of crossing durations across all studied processe.
\end{enumerate}
To summarize, the overall evidence is in favour of the conjecture, it is absolutely
necessary to address the issue of the upward bias in the empirical probabilities of
crossings of size higher than 2.

One of the way in which this study can be improved is the generation of samples paths
of Hermite stochastic processes. Another line of inquiry in the topic of crossing trees
if their application to machine learning and generalization to higher dimensions, which
would find application in image processing. The third possible extension is the utilization
of the crossing tree, or its modification, in the problem of early detection of structural
breaks in univariate or multivariate time series data.

% chapter conlusion_and_further_work (end)